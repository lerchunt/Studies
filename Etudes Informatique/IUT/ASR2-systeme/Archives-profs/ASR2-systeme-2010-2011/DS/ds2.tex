% rubber: rules ./rules.ini
\documentclass[11pt]{article}
\usepackage{a4wide}
\usepackage{styles/sujet_iut}

\usepackage{multicol}
\usepackage{moreverb}
\usepackage{vmargin}
\usepackage{parskip}
\usepackage{pifont}

\titreEpreuve{ASR2 Système\\\vspace{0.2cm}
Devoir Surveillé 2}
\dateEpreuve{Mercredi 15 juin 2011}
\publicEpreuve{Semestre 2}

\setmarginsrb{2cm}{0.5cm}{3cm}{1cm}{0cm}{1cm}{0cm}{0cm}
\newcounter{cq}
\setcounter{cq}{1}

\begin{document}
\maketitle

\vspace{0.2cm}
Sans documents, 45 minutes, \textbf{marquez votre nom en haut de la feuille}. 
\underline{Justifiez vos réponses}. 

%%%%%%%%%%%%%%%%%%%%%%%%
\vspace{-0.4cm}
\section{Questions de cours (\emph{2 points})}
\vspace{-0.4cm}
Qu'est ce qu'un système d'exploitation~? Listez les principaux mécanismes de gestion
mis en \oe{}uvre par les systèmes d'exploitation. 
\boite{\thecq.\stepcounter{cq}}{5.5}

%%%%%%%%%%%%%%%%%%%%%%%%%
\vspace{-0.4cm}
\section{Pagination à la demande (\emph{7 points})}
\vspace{-0.4cm}
On suppose une mémoire paginée avec de 3 cadres de pages. Les pages,
numérotées 0,1, ... sont appelées selon la séquence : \texttt{0, 1, 2,
  3, 4, 1, 2, 4, 0, 1}\\
On étudie l'effet de différents algorithmes de remplacement de page.

\vspace{-0.4cm}
\subsection{Algorithme FIFO}
\vspace{-0.4cm}
Utilisez le tableau suivant pour montrer le déroulement de l'algorithme FIFO.

\begin{center}
\thecq.\stepcounter{cq} 
\begin{tabular}{|c|c|c|c|c|c|c|c|c|c|c|}
\hline
Page & 0& 1& 2& 3& 4& 1& 2& 4& 0& 1 \\

\hline \hline
Cadre 1 &&&&&&&&&& \\ \hline
Cadre 2 &&&&&&&&&& \\ \hline
Cadre 3 &&&&&&&&&& \\ \hline
\hline
Défauts &&&&&&&&&& \\ \hline
\end{tabular}\\
Nombre de défauts de page =  
\end{center}

\vspace{-0.4cm}
\subsection{Algorithme LRU}
\vspace{-0.4cm}
Expliquez rapidement le fonctionnement de l'algorithme LRU
\boite{\thecq.\stepcounter{cq}}{3.5}
\newpage
Montrez le déroulement de cet algorithme sur le tableau suivant.
\begin{center}
\thecq.\stepcounter{cq}
\begin{tabular}{|c|c|c|c|c|c|c|c|c|c|c|}
\hline
Page & 0& 1& 2& 3& 4& 1& 2& 4& 0& 1 \\
\hline \hline
Cadre 1 &&&&&&&&&& \\ \hline
Cadre 2 &&&&&&&&&& \\ \hline
Cadre 3 &&&&&&&&&& \\ \hline
\hline
Défauts &&&&&&&&&& \\ \hline
\end{tabular}\\
Nombre de défauts de page =  
\end{center}

\vspace{-0.4cm}
\subsection{Comparaison stratégies LRU et FIFO}
\vspace{-0.4cm}
Pourquoi la stratégie LRU est elle \emph{en pratique} meilleure que FIFO ?
\boite{\thecq.\stepcounter{cq}}{3}

% PRINCIPE DE LOCALITÉ : les pages qui ont été utilisés récemment ont une probabilité
%    forte d'être réutilisées à court terme, par rapport aux pages "quelconques".

Construisez un scénario pathologique pour lequel LRU fait \textbf{plus de
défauts de page} que FIFO. Utilisez un contre-exemple simple avec 2 cadres, 3
pages, et une séquence de 5 références (donnez en la séquence de références de pages
et montrez le déroulement dans les deux cas).
\boite{\thecq.\stepcounter{cq}}{5}

%% les deux déroulement 
%%    reponse: 0,1,0,2,1
%%    0,1 les deux première étapes remplissent les cadres
%%    0 fait la différence entre les deux algos
%%    2 chasse la page 0 ou la page 1, selon l'algo
%%    1 charge la page qui vient d'être éjectée par lRU,  pour l'embeter


%%%%%%%%%%%%%%%%%%%%%%%%%%%%%%%
\vspace{-0.4cm}
\section{Systèmes RAID (\emph{4 points})}\vspace{-0.4cm}
\vspace{-0.3cm}

On dispose de 3 disques de 1 T octet montés en RAID 5.\\
Quelle est la capacité du système de disques ainsi constitué~? Comment sont répartis les blocs 0,1,2,3,4... ?
Faîtes un schéma. 
\boite{\thecq.\stepcounter{cq}}{6}
\newpage
Quelles opérations d'entrées-sorties faut-il faire sur les différents disques pour écrire dans le bloc 1 ? Précisez quelles opérations peuvent être menées
simultanément.
\boite{\thecq.\stepcounter{cq}}{4}

\vspace{-0.4cm}
\section{Accès disque, déplacement de bras (\emph{3 points})}
\vspace{-0.4cm}
Donnez les 3 algorithmes de déplacement de bras présentés en cours. 
Pour chacun, explicitez la stratégie employée et donnez ses avantages et inconvénients.
\boite{\thecq.\stepcounter{cq} Algorithme~:\vspace{1.2cm}\\
  \vspace{1.2cm}~~~Avantages~:\\
  \vspace{1.2cm}~~~Inconvénients~:}{0.7}
\boite{\thecq.\stepcounter{cq} Algorithme~:\vspace{1.2cm}\\
  \vspace{1.2cm}~~~Avantages~:\\
  \vspace{1.2cm}~~~Inconvénients~:}{0.7}
\boite{\thecq.\stepcounter{cq} Algorithme~:\vspace{1.2cm}\\
  \vspace{1.2cm}~~~Avantages~:\\
  \vspace{1.2cm}~~~Inconvénients~:}{0.5}
  
\vspace{-0.4cm}
\section{Adressage (4 points)}
\vspace{-0.4cm}

Considérons un système de mémoire virtuelle ayant les caractéristiques 
suivantes :
\begin{itemize}
\item taille d'une page et d'un cadre de page (ou bloc ou case) = 1Ko
\item taille de la mémoire physique (ou vive ou principale) = 32 Mo
\item taille de la mémoire virtuelle = 512 Mo,
\item utilisation combinée des techniques de pagination et segmentation~: 
\begin{itemize}
\item l'espace d'adressage virtuel d'un processus est composé de segments contigus,
\item chaque segment peut contenir entre une et 128 pages,
\item la numérotation des pages d'un segment est relative au segment.\\
\end{itemize}
\end{itemize}

Donnez le format d'une adresse virtuelle (nombre de bits des numéros
de segment, de page, et du déplacement)
\boite{\thecq.\stepcounter{cq} (Justifiez votre réponse)}{6}

Donnez le format d'une adresse réelle (nombre de bits du numéro de bloc et du déplacement
\boite{\thecq.\stepcounter{cq} (Justifiez votre réponse)}{6}

\end{document}

