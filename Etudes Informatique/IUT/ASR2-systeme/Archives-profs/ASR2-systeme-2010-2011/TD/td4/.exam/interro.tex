\documentclass[11pt]{article}

\usepackage{a4wide}
\usepackage{sujets}
\usepackage{verbatim}
\usepackage{moreverb}

\quand{Semestre 2, ann�e 2008-2009}
%\siglefeuille{DS}
\siglemat{ASR2}
\titrefeuille{Contr�le de Connaissances}
\formation{D�partement Informatique IUT Bordeaux 1}
\titre{Syst�mes d'exploitation}

\sloppy
\begin{document}

\section{Questions de cours}
\begin{enumerate}
\item Citez deux contraintes des syst�mes embarqu�s.
\item Citez les deux branches principales de la famille Unix
\end{enumerate}

\section{Ordonnancement}
On consid�re une machine sur laquelle vont tourner les 5 processus A, B,
C, D et E dont les dur�es d'ex�cution et les dates d'arriv�e sont les
suivantes :

\begin{center}
\begin{tabular}{|r||cc|}
\hline
processus & Date d'arriv�e & Dur�e\\
\hline
A&0&10\\
B&2& 4\\	
C&1& 7\\	
D&4&12\\
E&3& 5\\
\hline
\end{tabular}
\end{center}
	
On appelle temps de service (TS) d'un processus le temps �coul� entre
son arriv�e dans le syst�me et sa terminaison.

Repr�sentez le d�roulement de l'ex�cution des processus par un
diagramme et calculez le temps de service moyen pour les algorithmes
d'ordonnancement suivant :
\begin{enumerate}
\item Ordonnancement FIFO sans quantum de temps
\item Ordonnancement suivant la r�gle dite du "Tourniquet" (Round
 Robin) avec quantum de temps q=2 en tenant compte des dates
 d'arriv�e.
\end{enumerate}

Remarque : lorsqu'un processus termine son quantum de temps ``t'', il
est plac� dans la file d'attente AVANT un processus qui arrive au m�me
temps ``t''

\section{Memoire}
 Pour chacune des adresses virtuelles suivantes, donnez le num�ro de
 page virtuelle et le d�placement pour des pages de 4~ko et de 8~ko :
 0xbf7e72a0, 0x51a5f0c8


%% \vspace{2cm}
%% Version : 
%% \verb+$Id: ds-asr2-mai2007-v1.tex,v 1.2 2007/05/15 14:48:49 billaud Exp billaud $+




\end{document}
