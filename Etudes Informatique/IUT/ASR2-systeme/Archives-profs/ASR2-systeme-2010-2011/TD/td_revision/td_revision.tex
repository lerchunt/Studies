% rubber: rules ../rules.ini
\documentclass[10pt,twoside]{article}

\usepackage{a4wide}
\usepackage[T1]{fontenc}
\usepackage[latin1]{inputenc}
\usepackage[french]{babel}
\usepackage{a4wide}
\usepackage{mdwlist}
\usepackage{xcolor}
\usepackage{verbatim}
\usepackage{xspace}
\usepackage{graphicx}
\usepackage{subfigure}
\usepackage{vmargin}

% c pour afficher le corriger, nc sinon
\usepackage[nc]{optional}

\definecolor{grey2}{rgb}{0.92,0.92,0.95}
\newcommand\corr[1]{%
  \opt{c}{
    \fcolorbox{grey2}{grey2}{
      \begin{minipage}{0.9\textwidth}
        #1
    \end{minipage}}
  }
}

%\setmarginsrb{G}{H}{D}{B}{en-t�te}{dist-ent�te-texte}{pied}{dist_pied-texte}
\setmarginsrb{2cm}{2cm}{3cm}{1cm}{1cm}{0cm}{0cm}{1cm}
%\addtolength{\textwidth}{0cm} 

\sloppy
\begin{document}
\date{}
\title{R�visions 1}
\maketitle


\section*{Exercice 1}

Un syst�me de fichier utilise des blocs disque de 2~Ko. La taille m�diale d'un
fichier est de 1~Ko. Si tous les fichiers sont exactement de 1~Ko quel sera le
pourcentage de l'espace disque qui sera gaspill�? Pour un syst�me de fichiers r�el
pensez vous  que le pourcentage de gaspillage sera sup�rieur ou inf�rieur � cette
valeur? Expliquez

\section*{Annales 2009~: Gestion des entr�es-sorties}

Soient 3 processus P1, P2 et P3 arriv�s dans cet ordre � peu
pr�s au m�me instant sur un syst�me mutit�che. Ils se comportent tous les
3 de fa�on cyclique. P1 fait du calcul pendant un temps t, puis lit le bloc 100, puis
calcule, lit le bloc 101, calcule, lit le bloc 102, etc.  Il en est de m�me pour P2
et P3, qui font leurs lectures � partir des blocs 500 et 300.

Dans tout l'exercice, l'ordonnancement des processus est fait par
l'algorithme du tourniquet.  Le temps de r�alisation d'une
    entr�e/sortie d�pend de la position de l'entr�e sortie
    pr�c�dente. On prendra la formule

\begin{verbatim}
            dur�e = 5 + (depl/10) ms
\end{verbatim}
o� \texttt{depl} est le d�placement n�cessaire, en nombre de blocs. Par exemple la
lecture du bloc 123 apr�s celle du bloc 12, prendra \texttt{5+(123-12)/10 = 16 ms}.
La division est arrondie.
Initialement la t�te de lecture est en position 0.

\begin{enumerate}
\item Calculez le temps n�cessaire pour chacune des transitions
\begin{verbatim}
     t0: 0->100
     t1: 100->500
     t2: 500->300
     t3: 101->501
     t'1; 501->301
\end{verbatim}

\item �tudiez les 100 premi�res ms du comportement de l'ordonnancement
  "FIFO"pour les E/S sur disque, quand le temps de calcul est de t=3 ms.
  
%% \item Du point de vue des E/S, il n'y a gu�re de changement si le
%%   temps de calcul \texttt{t} passe de 3 � 4 ou 5 ms. Il en est autrement si il
%%   devient tr�s grand.
%%   Que se passe-t-il aux alentours du seuil
%%   \texttt{t = (t1+t2+t3) / 3}
  
\item �tudiez les 100 premi�res ms du comportement de l'ordonnancement
  "plus court d�placement "pour les E/S sur disque, quand le temps
  de calcul est de t=3 ms. Que se passe-t-il ensuite ?
\item Ce comportement "anormal" change quand \texttt{t} d�passe un certain seuil.
  Lequel, pourquoi ?
\end{enumerate}




\end{document}








% LocalWords:  rubber rules entr�es-sorties mutit�che depl
