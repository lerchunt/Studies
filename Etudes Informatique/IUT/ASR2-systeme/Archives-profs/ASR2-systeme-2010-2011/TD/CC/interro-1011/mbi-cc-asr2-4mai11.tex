\documentclass[10pt]{article}
\usepackage{a4wide}
\usepackage{sujets}

\usepackage{multicol}
\usepackage{moreverb}

\quand{Semestre 2, ann�e 2010-2011}
\siglefeuille{ASR2}
\siglemat{CC}
\titrefeuille{ASR2 Syst�me}
\formation{D�partement Informatique IUT Bordeaux 1}
\titre{Contr�le continu}

\newcommand{\boite}[1]
{
\noindent
\fbox{
  \begin{tabular}{l}
    #1 \\
    \hspace{0.95\linewidth}  \\
  \end{tabular}
}
}


% \sloppy

\begin{document}
Sans documents, 30 minutes, \textbf{marquez votre nom en haut de la feuille}.

\section{Etat des processus}

D�finir les �tats

\boite{1. Actif : \vspace{2cm}}

\boite{2. Pr�t : \vspace{2cm}}

\boite{3. Bloqu� : \vspace{2cm}}


Quand un processus peut-il passer 

\boite{4. de l'�tat pr�t � l'�tat actif ? \vspace{2cm}}

\boite{5. de l'�tat actif � l'�tat pr�t ? \vspace{2cm}}

\section{Ordonnancement - Syst�me non pr�emptif}

On consid�re les huit processus suivants :

\begin{center}
\begin{tabular}{c c c c}
\hline
Processus  & Instant d'arriv�e & Dur�e & Priorit� \\
\hline
P1 & 0  & 3  & 1 \\
P2 & 1  & 24 & 2 \\
P3 & 1  & 8  & 3 \\
P4 & 7  & 5  & 3 \\
P5 & 8  & 4  & 2 \\
P6 & 10 & 2  & 5 \\
P7 & 15 & 7  & 5 \\
P8 & 16 & 2  & 3 \\
\hline
\end{tabular}
\end{center}

%\newpage
Donnez l'ordre d'ex�cution des processus pour la politique
d'ordonnancement FIFO sans priorit�s. Calculez le temps de traitement moyen et le
temps de traitement maximal.\\ 
\boite{6.\vspace{2.5cm}}

Idem pour la politique d'ordonnancement FIFO avec priorit�s (sans
r�quisition). La priorit� 5 est la plus urgente. \\
\boite{7.\vspace{2.5cm}}

Idem pour la politique d'ordonnancement Plus Court Temps
d'Ex�cution. \\
\boite{8. \vspace{2.5cm}}
 


\end{document}
