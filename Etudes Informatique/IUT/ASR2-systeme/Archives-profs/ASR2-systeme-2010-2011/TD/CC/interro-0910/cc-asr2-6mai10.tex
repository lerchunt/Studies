\documentclass[12pt]{article}
\usepackage{a4wide}
\usepackage{sujets}

\usepackage{multicol}
\usepackage{moreverb}

\quand{Semestre 2, ann�e 2009-2019}
\siglefeuille{ASR2}
\siglemat{CC}
\titrefeuille{ASR2 Syst�me}
\formation{D�partement Informatique IUT Bordeaux 1}
\titre{Contr�le continu}

\newcommand{\boite}[1]
{
\noindent
\fbox{
  \begin{tabular}{l}
    #1 \\
    \hspace{0.95\linewidth}  \\
  \end{tabular}
}
}


% \sloppy

\begin{document}
Sans documents, 25 minutes, \textbf{marquez votre nom en haut de la feuille}.

%% \section{Innovations}

%% Les noms suivants ont marqu� des �tapes importantes dans l'histoire de l'informatique. A quoi vous font-il penser ?

%% \boite{1. MULTICS : \vspace{3cm}}

%% \boite{2. PDP/1 : \vspace{3cm}}

%% \boite{3. IBM/360  : \vspace{3cm}}

%% \section{Les g�n�rations}

%% On distingue habituellement 4 ``g�n�rations'' d'ordinateurs et de
%% syst�mes d'exploitation. Explicitez

%% \boite{4. Premi�re g�n�ration \vspace{3cm}}
%% \boite{5. Seconde g�n�ration \vspace{3cm}}
%% \boite{6. Troisi�me g�n�ration \vspace{3cm}}
%% \boite{7. Quatri�me g�n�ration \vspace{3cm}}

\section{Syst�me d'exploitation}

Qu'est ce qu'un syst�me d'exploitation?
Que g�re t-il?

\boite{1. \vspace{3cm}}
 
\section{Processus}

Qu'est ce qu'un processus?

\boite{2.\vspace{2cm}}

Qu'est-ce qu'un PCB, que contient-il ?

\boite{3. \vspace{2.5cm}}

Nommez et d�crivez les trois �tats possibles d'un processus

\boite{4.  \vspace{2.5cm}}
\boite{5.  \vspace{2.5cm}}
\boite{6.  \vspace{2.5cm}}

Comment se nomment les deux types de syst�mes multi-t�ches ?

\boite{7.  \vspace{2.5cm}}


\section{Ordonnancement}

D�finir en quelques mots ce qu'est l'ordonnancement

\boite{8. \vspace{3cm}}

Citez deux strat�gies d'ordonnancement. D�crivez rapidement leur principe.

\boite{9. \vspace{5cm}}

Les interruptions peuvent �tre de diff�rents types. Par exemple,
celles caus�es par un p�riph�rique qui signale la fin d'une
op�ration. Citez deux autres types.

\boite{10. \vspace{3cm}}
\end{document}

