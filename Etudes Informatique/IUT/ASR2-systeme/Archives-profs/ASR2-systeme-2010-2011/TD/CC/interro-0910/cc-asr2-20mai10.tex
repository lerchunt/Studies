\documentclass[12pt]{article}
\usepackage{a4wide}
\usepackage{sujets}

\usepackage{multicol}
\usepackage{moreverb}

\quand{Semestre 2, ann�e 2009-2019}
\siglefeuille{ASR2}
\siglemat{CC}
\titrefeuille{ASR2 Syst�me}
\formation{D�partement Informatique IUT Bordeaux 1}
\titre{Contr�le continu}

\newcommand{\boite}[1]
{
\noindent
\fbox{
  \begin{tabular}{l}
    #1 \\
    \hspace{0.95\linewidth}  \\
  \end{tabular}
}
}


% \sloppy

\begin{document}
Sans documents, 35 minutes, \textbf{marquez votre nom en haut de la feuille}.

\section{Processus}
Le PCB (\emph{Process Control Block}) contient le contexte d'un
processus. Qu'est ce que le contexte d'un processus?\\
\boite{1. \vspace{2.5cm}}

\section{Ordonnancement - Syst�me non pr�emptif}

On consid�re les huit processus suivants :

\begin{center}
\begin{tabular}{c c c c}
Processus  & Instant d'arriv�e & Dur�e & Priorit� \\
\hline
P1 & 0  & 3  & 1 \\
P2 & 1  & 24 & 2 \\
P3 & 1  & 8  & 3 \\
P4 & 7  & 5  & 3 \\
P5 & 8  & 4  & 2 \\
P6 & 10 & 2  & 5 \\
P7 & 15 & 7  & 5 \\
P8 & 16 & 2  & 3 \\
\end{tabular}
\end{center}

%\newpage
Donnez l'ordre d'ex�cution des processus pour la politique
d'ordonnancement FIFO. Calculez le temps de traitement moyen et le
temps de traitement maximal.\\ 
\boite{2.\vspace{2.5cm}}

Idem pour la politique d'ordonnancement FIFO avec priorit�s (sans
r�quisition). \\
\boite{3.\vspace{2.5cm}}

Idem pour la politique d'ordonnancement Plus Court Temps
d'Ex�cution. \\
\boite{4. \vspace{2.5cm}}
 
\section{M�moire}

D�crivez le principe de cohabitation en m�moire contigu�.\\
Quels probl�mes cela pose t-il? \\
\boite{5.\vspace{2cm}}


Expliquez la diff�rence entre adresse logique et adresse physique?\\
\boite{6.\vspace{3cm}}

Expliquez le principe de m�moire pagin�e? \\
\boite{7.  \vspace{2.5cm}}

A quoi sert une table des pages?\\
\boite{8.  \vspace{2.5cm}}

Qu'est ce que la MMU? Quel est son r�le?\\
\boite{9.  \vspace{2.5cm}}\\

\newpage
\noindent \textbf{Remplacement de page}\\ \\
Soit la liste des pages virtuelles r�f�renc�es aux instants t = 1, 2,
..., 10 : \\
  0 1 7 2 3 2 7 1 0 3 \\
La m�moire est compos�e de 4 cases initialement vides.
Repr�sentez l'�volution de la m�moire au fur et � mesure des acc�s
pour chacune des deux politiques de remplacement FIFO et LRU.
Notez le nombre de d�fauts de page pour chaque cas.

\boite{10.  \vspace{6cm}}
\boite{11.  \vspace{6cm}}


\end{document}


% LocalWords:  Process Control Block pr�emptif FIFO pensez-vous t-il MMU LRU
