\def\col{blue}
\documentclass[14pt, \col, hyperref={pdfpagelabels=false}]{beamer}
\usepackage[frenchb]{babel} % langue francaise
\usepackage[utf8]{inputenc} % encodage UTF-8
\usepackage[T1]{fontenc} % encodage de la police de caracteres
\usepackage{lmodern}
\usepackage{listings}
\usepackage{multicol}
\usepackage{pifont} %\ding{number}, 
\usepackage{subfigure}
\usepackage{graphicx}
\usepackage{listings}

\usetheme{Warsaw}
\usepackage[\col]{optional} %red blue green
%% THÈMES couleur
%% \usecolortheme{dolphin}
%% \usecolortheme{seahorse}
%% * thèmes globaux : beetle, crane, fly, seagull.
%% * thèmes internes : lily (enlève surtout des couleurs), orchid, rose.
%% * thèmes externes : whale, seahorse, dolphin

% rubber: rules ./rules.ini

 
\mode<presentation>
{
  %%%% Couleurs
  \definecolor{red2}{rgb}{0.8,0.15,0.15}
  \definecolor{bx}{rgb}{0.45,0.05,0.05}  
  \definecolor{ivoire}{rgb}{1, 0.97, 0.9}
 
  \definecolor{bleu2}{rgb}{0.9, 0.90, 1}
  \definecolor{bleu3}{rgb}{0.15,0.15,0.7}
  \definecolor{myblue}{rgb}{0.50,0.55, 0.8}  
  \definecolor{myblue2}{rgb}{0.25,0.3, 0.65}  
  
 \definecolor{mygreen2}{rgb}{0.15,0.60,0.15}
 \definecolor{green2}{rgb}{0.15,0.40,0.15}
  \definecolor{mygreen}{rgb}{0.40,0.7, 0.4}  
%  \definecolor{lightgreen}{rgb}{0.5,0.8, 0.5}  
  \definecolor{lightgreen}{rgb}{0.95, 1, 98}
  
  \definecolor{medium-grey}{gray}{0.45}
  \definecolor{light-grey}{gray}{0.85}
  
  % Couleur des structures et fond dégradé
  % defaut: green
  \setbeamercolor{normal text}{fg=black,bg=lightgreen}
  \setbeamercolor{structure}{fg=green2, bg=light-grey} 
  \setbeamercolor{alerted text}{fg=green2}
  \setbeamertemplate{background canvas}[vertical
    shading][top=lightgreen!100,bottom=lightgreen!30]
  
  \opt{blue}{
    \setbeamercolor{normal text}{fg=black,bg=bleu2}
    \setbeamercolor{structure}{fg=myblue, bg=light-grey} 
    \setbeamercolor{alerted text}{fg=myblue2}
    \setbeamertemplate{background canvas}[vertical
      shading][top=bleu2!100,bottom=bleu2!30]
  }
  \opt{red}{
    \setbeamercolor{normal text}{fg=black,bg=ivoire}
    \setbeamercolor{structure}{fg=bx, bg=light-grey} 
    \setbeamercolor{alerted text}{fg=red2}
    \setbeamertemplate{background canvas}[vertical
      shading][top=ivoire!40,bottom=ivoire!100]
  }
  \opt{green}{
    \definecolor{green2}{rgb}{0.1,0.60,0.1}
    \usecolortheme{dolphin}
  }   
  
  
  \setbeamertemplate{navigation symbols}{}
  \setbeamerfont{title in head/foot}{size=\scriptsize}
  \setbeamerfont{date in head/foot}{size=\scriptsize} 
  
  % Pied de page
  \setbeamertemplate{footline}{
    \hbox{%
      \begin{beamercolorbox}[wd=.4\paperwidth,ht=3ex,dp=1ex,center]{title in head/foot}
        \usebeamerfont{title in head/foot}\enseignement\hspace{1cm} %\shorttitle
      \end{beamercolorbox}%
      \begin{beamercolorbox}[wd=.5\paperwidth,ht=3ex,dp=1ex,left]{date in head/foot}%
        \usebeamerfont{title in head/foot} 
        \hspace*{1ex} 
        \insertshorttitle
      \end{beamercolorbox}% 
      \begin{beamercolorbox}[wd=.1\paperwidth,ht=3ex,dp=1ex,left]{date in head/foot}%
        \usebeamerfont{date in head/foot} 
        \insertframenumber{} / \inserttotalframenumber%\hspace*{2ex} 
    \end{beamercolorbox}}%
    \vskip0pt%
  }

    % Haut de page
  \setbeamertemplate{headline}{%
    \leavevmode
    \begin{beamercolorbox}[wd=.5\paperwidth,ht=3ex,dp=1.125ex,leftskip=.3cm
        plus1fill,rightskip=.3cm]{section in head/foot}%
      \scriptsize{\insertsection}
    \end{beamercolorbox}%
    \begin{beamercolorbox}[wd=.5\paperwidth,ht=3ex,dp=1.125ex,leftskip=.3cm,rightskip=.3cm
        plus1fil]{subsection in head/foot}%
      \scriptsize{\insertsubsection}
    \end{beamercolorbox}%
  }
  \addtobeamertemplate{headline}{}{%
    \vskip-0.1pt
    \pgfuseshading{beamer@topshade}
    \vskip-2pt}
   %% \let\Tiny=\tiny
  %% \let\TINY=\tiny
}

%% Annonce de plan a chaque transition.
%% \AtBeginSection[]{
%%   \begin{frame}<beamer>
%%     \frametitle{Plan}
%%     \tableofcontents[currentsection,currentsubsection]
%%   \end{frame}
%% }



\newcommand{\bashlisting}[0]{\scriptsize\lstset{language=bash,numbers=left,numberstyle=\tiny,
    xrightmargin=1mm, xleftmargin=1mm, keywordstyle=\color{green2}, %
    keywordstyle[1]=\color{blue}, %
    keywordstyle[2]=\color{yellow}, %
    keywordstyle[3]=\color{red2}}}

\newcommand{\filelisting}[0]{\small\lstset{language=,numbers=left,numberstyle=\tiny,
    xrightmargin=4mm, xleftmargin=6mm, frame=single}}

\definecolor{orange}{rgb}{0.9,0.6,0.4}

\newcommand{\clisting}[0]{\scriptsize
  \lstset{language=c, 
    inputencoding=utf8,
    %identifierstyle=\color{blue},%
    keywordstyle=\color{green2}, %
    keywordstyle[1]=\color{blue}, %
    keywordstyle[2]=\color{yellow}, %
    keywordstyle[3]=\color{red2}, %
    stringstyle=\color{orange}, %
    commentstyle=\color{red2}, %
    numbers=left,
    numberstyle=\tiny,
    xleftmargin=1mm,
    breaklines=true,
}}



\newcommand{\cmd}[1]{\texttt{\footnotesize\textcolor{medium-grey}{#1}}}

\newcommand{\cmdlist}[1]{
  \begin{semiverbatim}\begin{minipage}{\linewidth}  
      \footnotesize\textcolor{medium-grey}{#1}\end{minipage}
\end{semiverbatim}}


\newcommand{\coeur}{c\oe ur\xspace}
\newcommand{\oeuvre}{\oe uvre\xspace}
\newcommand{\coeurs}{c\oe urs\xspace}
\newcommand{\mc}{multic\oe ur\xspace}
\newcommand{\mcs}{multic\oe urs\xspace}

\usepackage{multicol}

%%%%%%%%%%%%%
\def\enseignement{ASR2 Système}
\title[]{\enseignement\\
}
\author{Stéphanie Moreaud}
\institute{Département d'informatique\\
  IUT Bordeaux 1}
\date{}

%%%%%%%%%%%% 


\begin{document}

\begin{frame}
  \titlepage
\end{frame}

%%%%%%%%%%%%%%%%%%%%%%%
\begin{frame}<beamer>
  \frametitle{Plan}
  \tableofcontents[hideallsubsections]
\end{frame}
%%%%%%%%%%%%%%%%%%%%%%%%%

%Annonce de plan a chaque transition.
\AtBeginSection[]{
  \begin{frame}<beamer>
    \frametitle{Plan}
    \tableofcontents[currentsection, hideothersubsections]
  \end{frame}
}

\section{Présentation}

\begin{frame}
  \frametitle{Objectifs}
  Comprendre les principes de fonctionnement des systèmes d'exploitation
  multitâches et multi-utilisateurs, au niveau~:
  \begin{itemize}
  \item de l’utilisation
  \item de la structure interne 
  \item de la mise en œuvre
  \end{itemize}
\end{frame}

\begin{frame}
\frametitle{Organisation du module}
4 cours de 1h50\\
13 séances de TDs de 1h50\\
\vspace{0.5cm}
Évaluation~:
\begin{itemize}
\item contrôle continu~: une note de TD 
\item 2 notes de DS
\end{itemize}
\end{frame}

\begin{frame}
  \frametitle{Plan du cours}
  \begin{itemize}
  \item Introduction aux systèmes d'exploitation, notion de processus et ordonnancement
  \item Gestion de la mémoire
  \item Stockage des données et système de fichiers.
  \item Virtualisation, tendances...
  \end{itemize}
  \end{frame}

\begin{frame}
\frametitle{Bibliographie}
\bibliographystyle{fr-plain}
\small
\nocite{*}    
\bibliography{bib}  
\vspace{0.5cm}    
Remerciements à Michel Billaud et Ludovic Courtès
\end{frame}

%%%%%%%%%%%%%%%%%% OS rôle et définitions  %%%%%%%%%%%%%%%%%%%%%%%%%%
\section{Introduction aux systèmes d'exploitation}

\subsection{Définitions}
\begin{frame}
\frametitle{Qu'est ce qu'un système d'exploitation~?}
{Définition}~:\alert{un Système d'Exploitation} (\emph{Operating System}) est une
\alert{couche d'abstraction} (ensemble de programmes) construite au dessus du matériel pour
\begin{itemize}
\item masquer la complexité du matériel
\item arbitrer l'accès aux ressources
\end{itemize}
\begin{figure}[h]
  \begin{minipage}[t]{\linewidth}
    \center
    \includegraphics[width=0.5\textwidth]{fig/OS}
  \end{minipage}
\end{figure}
\end{frame}


%%%%%
\subsection{Rôle d’un système d’exploitation}
\begin{frame}
\frametitle{\insertsubsection}

Le système d'exploitation doit permettre~:
\begin{itemize}
\item d'\alert{exploiter les ressources matérielles} efficacement
\item de fournir aux programmeurs d'applications un \alert{environnement}
  (\alert{machine étendue} ou \alert{virtuelle}) plus simple à programmer que la
  \emph{machine réelle}.
\end{itemize}
\vspace{0.5cm}

Il est notamment~: 
\begin{itemize}
\item support d'abstraction du matériel
\item piller de la gestion des ressources
\item garant de la sûreté de fonctionnement (sécurité, tolérance aux
  fautes, gestion des erreurs, etc.)
\end{itemize}
\end{frame}

\begin{frame}
  \frametitle{\insertsubsection} 
  Le système d'exploitation \alert{cache les détails matériels} 
  sous une \alert{couche d'abstraction.}
  \begin{itemize}
  \item communication réseau par différents moyens,
  \item utilisation de fichiers sur différents supports, ...
  \end{itemize}
  \vspace{0.5cm}
  Il gère l'utilisation des ressources  
  \begin{itemize}
  \item processeurs, mémoire, périphériques, ...
  \end{itemize}
  \vspace{0.5cm}
  Il veille à ce qu'un certains nombre d'\emph{opérations dangereuses} ne puissent
  être effectuées qu'avec des droits privilégiés.  
  \end{frame}

%%%%%%%%%%%%%%%%%%   Un peu d'histoire     %%%%%%%%%%%%%%%%%%%%%%%%%%
\section{Évolution des systèmes}

%%%%%%%%%%%%%%%%%  
\subsection{Évolution des systèmes}
\begin{frame}
  \frametitle{\insertsubsection}
  \alert{Première génération} (1945-1955)~: machines à tubes.
  \begin{itemize}
  \item manipulation d'interrupteurs à bascule 
  \item début des années 50, introduction des cartes perforées 
  \item une seule tâche à la fois, intervention de l'opérateur avant/après chaque
    travail      
    \begin{itemize}
    \item[\ding{212}] système \alert{monotâche}
    \end{itemize}
  \end{itemize}
  \vspace{0.5cm}
  \alert{Seconde génération} (1955-1965)~: machines à transistors
  \begin{itemize}
  \item création des systèmes de \emph{batch} (traitement par lots)
  \item enchaînement automatique d'un \emph{train de travaux}
    assuré par un programme spécial (ancêtre des SE)
    \begin{itemize}
    \item[\ding{212}] système à \alert{traitement par lots}
    \end{itemize}
  \end{itemize}
\end{frame}

\begin{frame}
  \frametitle{\insertsubsection}
  \alert{Troisième génération} (1965-1971)~: circuits intégrés
  \begin{itemize}
  \item exécution de plusieurs programmes chargés en mémoire
  \item réponse au blocage sur les périphériques  
  \item[\ding{212}] \alert{premiers systèmes d'exploitation} (OS/360, 1966), \alert{multiprogrammation}
  \end{itemize}
  \vspace{0.5cm}  
  \begin{itemize}
  \item réponse à la demande d'interactivité des utilisateurs 
  \item[\ding{212}] généralisation des systèmes à \alert{temps partagé}
    \begin{itemize}
    \item introduit en 1962 par le CTSS (\emph{Compatible Time Sharing System}) du MIT 
    \end{itemize}
\end{itemize}
\end{frame}

\begin{frame}
\frametitle{\insertsubsection}
\begin{itemize}
  \item utilisation du principe de multiprogrammation avec un passage d'un programme à l'autre après
    quelques millisecondes
  \item[\ding{212}] système \alert{multitâche}
    \begin{itemize}
    \item illusion que les programmes s'exécutent simultanément
    \item Multics (MIT, 1965), premier système multitâche multi-utilisateurs 
    \item UNIX créé en 1970 sur les bases de Multics, facile à porter 
    \end{itemize}
  \end{itemize}
\end{frame}
  
\begin{frame}
  \frametitle{\insertsubsection}
  \alert{Quatrième génération} (1972-aujourd'hui)~: micro-ordinateurs
  \vspace{0.2cm}
  \begin{itemize}
    \item création de systèmes d'exploitation adaptés aux premiers micro-ordinateurs
      \begin{itemize}
      \item CP/M, QDOS, MS-DOS
      \end{itemize}
      \vspace{0.2cm}
    \item intégration bonus des premières interfaces graphiques  
      \begin{itemize}
      \item Macintosh \emph{System} (1984),
      \item environnement Windows sur MS-DOS
      \item etc. 
      \end{itemize}
  \end{itemize}
\end{frame}

\subsection{Bilan~: types de systèmes}
\begin{frame}
  \frametitle{\insertsubsection}
  L'évolution du matériel et des besoins a donné lieu à différents types de systèmes~:
  \begin{itemize}
  \item \alert{monotâche} (ou \emph{monoprogrammation})~: un programme à la fois,
    intervention d'un opérateur entre chaque traitement 
  \item \alert{monotâche à traitement par lots}~: enchaînement automatique d'un \emph{train de travaux}
    assuré par un programme.
  \item \alert{multiprogrammation} : plusieurs programmes en mémoire,
    profitent des temps morts pour s'exécuter.
  \item \alert{multitâche, temps-partagé} : plusieurs programmes en mémoire exécutés
    quantum par quantum, donne l'illusion de l'exécution simultanée
  \end{itemize}
\end{frame}



%%%%%%%%%%%%%% Qu'en est-il aujourd'hui~? 
\subsection{Situation actuelle}

\begin{frame}
  \frametitle{Qu'en est-il aujourd'hui~?}
  Interconnexion d'ordinateurs (réseaux)\\
  Poursuite de la miniaturisation
  \begin{itemize}
  \item généralisation du \alert{\mc} 
  \item[\ding{212}] intégration de plusieurs processeurs sur une même puce
  \end{itemize}
  Multiplications des ressources
  \begin{itemize}
  \item \coeurs, périphériques, etc. 
  \end{itemize}
  \alert{Systèmes embarqués} (téléphones, agendas, ...) 
  \begin{itemize}
  \item fortes contraintes matérielles
  \end{itemize}
\vspace{0.3cm}
\begin{itemize}
\item[\ding{212}] Complexification des aspects gérés par les systèmes d'exploitation
\item[\ding{212}] Différents types de systèmes d'exploitation pour répondre à des besoins précis
\begin{itemize}
\item serveurs, multiprocesseurs, personnels, distribués, temps réel, embarqués, etc.  
\end{itemize}
\end{itemize}
\end{frame}

\section{Composition du système d'exploitation}
\begin{frame}
\frametitle{\insertsection}
Les systèmes d'exploitations actuels comprennent généralement~:
\begin{itemize}
\item un \alert{noyau}
\begin{itemize}
\item[\ding{212}] proche du matériel
\item[\ding{212}] modulaire sur beaucoup de systèmes
\end{itemize}
\item des \alert{bibliothèques}
  \begin{itemize}
  \item[\ding{212}] code factorisé entre les applications
  \end{itemize}
\item des \alert{outils}
\begin{itemize}
  \item[\ding{212}] ensemble de programmes et de scripts
\end{itemize}
\end{itemize}
\vspace{0.5cm}
Il s'attache à la gestion de plusieurs éléments~:
\begin{itemize}
\item processus, mémoire, périphériques, fichiers, droits, informations, ...  
\end{itemize}
\end{frame}

%%%%%%%%%%%%%%%%%%%%%%%%%%%%%%%%%%%%%%%%%%%%%%%%%%%%%%

\subsection{Éléments gérés par le système d'exploitation}

\begin{frame}
\frametitle{Processus}
Le système assure le bon fonctionnement des \alert{processus} qui utilisent
du temps et de la mémoire
\begin{itemize}
\item \alert{partage de l'espace mémoire} disponible
\item \alert{répartition du temps} de fonctionnement
\item éventuelle attribution des processeurs
\item partage/\alert{protection} de la mémoire entre processus
\item terminaison de processus défaillant.
\end{itemize}
\end{frame}

\begin{frame}
\frametitle{Droits}
 Lors de leur enregistrement, les utilisateurs se voient accorder des
\alert{droits d'accès} limités, de même que les applications utilisateurs.\\
\vspace{0.5cm}

Le système d'exploitation fait respecter les limitations imposées.
\begin{itemize}
\item garantit que les ressources ne sont utilisées que par les programmes et
utilisateurs possédant les droits adéquats.
\end{itemize}
\end{frame}

\begin{frame}
  \frametitle{Périphériques et entrées-sorties}
  Le système d'exploitation prend en charge (unifie et contrôle) l'accès aux différentes ressources
  matérielles (périphériques) par le biais des \alert{pilotes} (\emph{drivers})
  \vspace{0.5cm}
  
  \underline{Définition}~: un pilote est un programme qui effectue les opérations de base
  pour \alert{un type de matériel donné}.
  \vspace{0.5cm}
  
  \underline{Exemples}~:
  \begin{itemize}
  \item lecture ou écriture d'un secteur sur un disque
  \item carte son, graphique, réseau
  \item ...
  \end{itemize}
\end{frame}

\begin{frame}
\frametitle{Fichiers}
\alert{Système de fichiers}~: 
\begin{itemize}
\item organisation arborescente (répertoires, sous-répertoires, ...)
\item stockage sur un \alert{support physique} (partition d'un disque, clé
    USB, CD, ...), ou accessibles par le réseau
  \item droits d'accès assurant la protection des données
  \item mécanisme de résistance aux pannes (journalisation, systèmes RAID, etc.)
\end{itemize}
\vspace{0.5cm}
Le système d'exploitation gère la lecture et l'écriture dans le système de fichiers
en tenant compte des droits d'accès aux fichiers des utilisateurs et applications.
\end{frame}


%%%%% Processus %%%%%%%%%%%%%%%%%%%%%%%%%%%%%%%%%%%%%%%%%%%%%%%%%%%%%%%%
\section{Généralités sur les processus}
\begin{frame}
\frametitle{\insertsection}
\vspace{0.5cm}
\underline{Définition}:
un processus est une entité abstraite qui représente
un \alert{programme en cours d'exécution}.\\
\vspace{0.6cm}

Un processus est défini par~:
\begin{itemize}
\item un \alert{ensemble d'instructions} (code du programme) 
\item un \alert{espace mémoire} pour les données de travail (pile, tas)
\item divers ressources
  \begin{itemize}
  \item descripteurs de fichiers ouverts, ports réseau, etc.
  \end{itemize}
\item des \alert{droits d'accès}
\end{itemize}
\vspace{0.5cm}
\end{frame}

\begin{frame}
  \frametitle{Espace mémoire d'un processus UNIX}
  \begin{minipage}[t]{\linewidth}
    \center
    \includegraphics[width=0.6\textwidth]{fig/espace_adressage_processus}
  \end{minipage}
\end{frame}



\begin{frame}
\frametitle{Modes de fonctionnement}
Il existe deux modes de fonctionnement~:
\begin{itemize}
\item le \alert{mode noyau} (privilégié)
  \begin{itemize}
  \item toutes les opérations sont autorisées 
  \end{itemize}
\item le \alert{mode utilisateur} (protégé)
  \begin{itemize}
  \item l'accès à la mémoire physique et aux périphériques est protégé
  \end{itemize}
\end{itemize}
\vspace{0.5cm}

Basculement du mode utilisateur vers le mode noyau~:
\begin{itemize}
\item sollicitation du système pour une opération particulière, \alert{appel système}
  \begin{itemize}
  \item[\small{ex:}] écriture dans un fichier, allocation mémoire, etc.
  \end{itemize}
\item interruption du système
  \begin{itemize}
  \item[\small{ex:}] suite à une erreur mémoire, un signal d'horloge, etc.
  \end{itemize}
\end{itemize}
\end{frame} 




\begin{frame}
\frametitle{États d'un processus}
3 états possibles: 
\begin{itemize}
\item \alert{actif}~: en cours d'exécution
\item \alert{bloqué}~: en attente d'un évènement 
\item \alert{prêt}~: ni actif, ni bloqué
\end{itemize}
\begin{minipage}[t]{\linewidth}
  \center
  \includegraphics[width=\textwidth]{fig/etats_processus_0}
\end{minipage}
\end{frame}


\subsection{Changements d'état}
\begin{frame}
  \frametitle{Changements d'états}
Un processus passe de l'état~:
\begin{itemize}
\item[\small(1)] \alert{actif} à \alert{bloqué} lorsqu'il demande une opération d'E/S.
\item[\small(2)] \alert{bloqué} à \alert{prêt} quand l'opération d'E/S est terminée
\item[\small(3)] \alert{prêt} à \alert{actif} lorsqu'il est choisi par le système 
\end{itemize}

\begin{minipage}[t]{\linewidth}
  \center
  \includegraphics[width=\textwidth]{fig/etats_processus_1}
\end{minipage}
\end{frame}

\begin{frame}
  \frametitle{Changements d'états}
Dans un système multitâche, plusieurs processus peuvent être prêts
\begin{itemize}
\item l'\alert{ordonnanceur} (\emph{scheduler}) est en charge de désigner le processus à exécuter.
\item différentes \alert{politiques d'ordonnancement} sont possibles.
\end{itemize}
\vspace{0.5cm}
\underline{Définition}~: l'\alert{ordonnanceur} est le composant du système qui
défini le prochain processus à activer.

\end{frame}

\subsection{Système multitâche préemptif / coopératif}
\begin{frame}
  \frametitle{\insertsubsection} 
  Pour obtenir un partage de temps processeur plus équitable, un processus actif doit
  pouvoir \emph{céder sa place}.
  \begin{itemize}
  \item exp~: monopolisation du processeur par un processus qui ne fait jamais d'E/S. 
  \end{itemize}
   \vspace{0.5cm}

  Pour cela, le \alert{processus actif} peut 
  \begin{itemize}
  \item \alert{rendre volontairement la main} 
  \item[\ding{212}] système \alert{coopératif}, sans réquisition
  \item \alert{être interrompu} au bout d'un \alert{quantum de temps} 
  \item[\ding{212}] système \alert{préemptif}, avec réquisition  
  \end{itemize}
\end{frame}

\begin{frame}
  \frametitle{\insertsubsection} 
  Pour obtenir un partage de temps processeur plus équitable, un processus actif doit
  pouvoir \emph{céder sa place}.
  \vspace{0.5cm}
  
  \begin{minipage}[t]{\linewidth}
    \center
    \includegraphics[width=\textwidth]{fig/etats_processus_2}
  \end{minipage}
\end{frame}


\subsection{Commutation du contexte}
\begin{frame}
  \frametitle{\insertsubsection}
  Le \alert{contexte} d'un processus contient toutes les informations relatives à
  l'exécution du processus (valeur du \alert{compteur ordinal}, des \alert{registres
    d'états}, du \alert{pointeur de pile}, allocations mémoire, etc.)\\
  \vspace{0.5cm}
  Lors des changements d'état, le contexte du processus~:
  \begin{itemize}
  \item qui était actif est \alert{sauvegardé} 
  \item qui devient actif est \alert{restauré}
  \item[\ding{212}] reprise de l'exécution où elle a été interrompue. 
  \end{itemize}
  \vspace{0.5cm}

\begin{itemize}
  \item[\tiny\ding{110}] sauvegarde du contexte~: structure de données appelée PCB (\alert{Process Control
    Block}), lors du basculement d'état.\\
  \item[\tiny\ding{110}] stockage des PCB des processus présents sur le système dans la \alert{table des processus}.
\end{itemize}
\end{frame}


%%%%%%%%%%%%%%%%%%%%%%%%%%%%%%%%%%%%%%%%%%%%%%%%%%%%
\section{Les interruptions}

\subsection{Notions d'interruption et système réactif}
\begin{frame}
\frametitle{\insertsubsection}
Le fonctionnement des systèmes d'exploitation contemporains s'appuie sur la notion
d'\alert{interruption}.\\
\vspace{0.5cm}

\alert{Interruption} : signal matériel qui modifie la séquence normale d'exécution
des instructions. \\
\vspace{0.2cm}
 
\begin{itemize}
\item[\ding{212}] \alert{système réactif}, répond aux évènements causés par l'environnement
\end{itemize}


Une interruption déclenche
\begin{itemize}
\item le passage en mode noyau
\item la \alert{sauvegarde de l'état} du programme (quelques registres)
\item l'exécution de la \alert{routine de traitement} de l'interruption.
\end{itemize}
\end{frame}

\subsection{Interruptions et système multitâche}
\begin{frame}
\frametitle{Dans un système multitâche}
  Les interruptions sont causées par~:
  \begin{itemize}
  \item les \alert{périphériques} (exp~: fin d'exécution de requête)
  \item des \alert{signaux d'horloge}
    \begin{itemize}
    \item réguliers, utilisés pour l'ordonnancement
    \end{itemize}
  \item des \alert{évènements extérieurs}
  \item des déroutements en cas d'\alert{erreur} (accès illégal à la mémoire, division par zéro, etc)
  \item des \alert{interruptions logicielles} provoquées par instruction spéciale
    \begin{itemize}
    \item sollicitation du noyau pour un service particulier
    \item[\ding{212}] \alert{appel système}
    \end{itemize}
  \end{itemize}
\end{frame}

\begin{frame}
\frametitle{\insertsubsection}
\begin{minipage}[t]{\linewidth}
  \center
  \includegraphics[width=\linewidth]{fig/interruption}
\end{minipage}
\end{frame}

\begin{frame}
\frametitle{Exemples}
Déroulement d'une interruption disque
\begin{enumerate}
  \item passer le processus actif à l'\alert{état prêt}
  \item déterminer la cause de l'interruption (ex~: fin de lecture)
  \item trouver le processus demandeur (bloqué)
  \item lui transférer les données reçues
  \item \alert{le mettre à l'état prêt}
  \item \alert{activer un des processus prêts}
\end{enumerate}
\vspace{0.5cm}
Interruption d'horloge
\begin{itemize}
\item quantum de temps épuisé \ding{212} ordonnancement avec réquisition
(\emph{preemptive scheduling})
\end{itemize}
\begin{enumerate}
  \item remettre le processus actif à l'\alert{état prêt}
  \item \alert{activer un des processus prêts}
\end{enumerate}
\end{frame}

\section{Ordonnancement}
\begin{frame}
\frametitle{\insertsection}
L'ordonnanceur à la charge de désigner le processus à exécuter (ordonnancer). 
\begin{itemize}
\item il consulte la liste des processus \alert{prêts} et en choisit un pour
  l'activation 
\end{itemize}

\begin{itemize}
\item[\ding{212}] applique une \alert{politique d'ordonnancement}
\end{itemize}
\vspace{0.5cm}

Comment définir/choisir une \alert{bonne politique} d'ordonnancement~?
\end{frame}

\begin{frame}
  \frametitle{\insertsubsection}
  Nombreux critères~:
  \begin{itemize}
  \item équité
  \item efficacité
  \item minimiser le temps de réponse
  \item minimiser le temps d'exécution
  \item maximiser le rendement, 
  \item maximiser l'occupation du/des processeur(s),
  \item  ...
  \end{itemize}
  \vspace{0.5cm}
  
  \begin{itemize}
  \item[\ding{212}] \alert{objectifs} parfois \alert{contradictoires}
  \item[\ding{212}] comportement des processus ne peut être prévu
  \item[\ding{212}] importance des critères fonction du contexte
  \end{itemize}
  \vspace{0.5cm}
  \ding{212} \alert{pas de politique optimale}, utilisation d'\alert{heuristiques}
\end{frame}

\subsection{Quelques stratégies}
\begin{frame}
  \frametitle{Ordonnancement circulaire}
  
  Autres dénominations~: tourniquet, round-robin, FIFO (\emph{Fisrt In First Out}), ...\\
  \underline{Principe~:}
  \begin{itemize}
  \item liste circulaire des processus prêts
  \item à la fin de son quantum de temps, un processus actif est placé
    en fin de liste, le premier élément de la liste est ordonnancé.  
  \end{itemize}
  
  \underline{Propriétés~:}
    \begin{itemize}
    \item \alert{équité} garantie
    \item choix du quantum~: compromis 
      \begin{itemize}
      \item trop court, nombreux changement de contexte, perte de temps
      \item trop long, dégradation du temps de réponse
      \end{itemize}
    \end{itemize}
\end{frame}

%\subsection{Priorités}
\begin{frame}
  \frametitle{Ordonnancement avec priorités}
  
  On affecte une priorité (numérique) à chaque processus.\\

  \underline{Principe~:}
  \begin{itemize}
  \item choix du processus le plus prioritaire
  \item dans la même classe de priorité~: FIFO
  \end{itemize}
  \vspace{0.5cm}

  \underline{Propriétés}
  \begin{itemize}
  \item risque de famine pour les classes de priorité inférieure
  \item[\ding{212}] remédier aux problèmes d'équité en faisant évoluer les priorités
    avec le temps (exp~: baisse en fin de quantum).
  \end{itemize}
\end{frame}

\begin{frame}
  \frametitle{Files multiples}
    On définit des \alert{classes} de processus\\
    \underline{Principe~:}
    \begin{itemize}
    \item à chaque classe correspond une liste (circulaire) de processus
    \item chaque classe est sélectionnée régulièrement
    \end{itemize}
    \vspace{0.2cm}
    \underline{Propriétés}
    \begin{itemize}
    \item respect des priorités
    \item évite les famines
    \end{itemize}

\vspace{0.5cm}
D'autres alogorithmes~: 
\begin{itemize}
\item Plus Court Temps d'Exécution
\item  Plus Court Temps Restant
\end{itemize}
\end{frame}

\end{document}



% LocalWords:  blue ASR beamer currentsection currentsubsection TDs DS Michel
% LocalWords:  multi-utilisateurs Billaud Ludovic Courtès Operating System PDP
% LocalWords:  width monotâche batch kilo-mots DEC d'interactivité CTSS Time CP
% LocalWords:  Sharing multitâche millisecondes Multics QDOS MS-DOS Windows USB
% LocalWords:  monoprogrammation temps-partagé entrées-sorties CD etats d'E exp
% LocalWords:  journalisation scheduler préemptif preemptive scheduling PCB
% LocalWords:  Process Control Block round-robin FIFO Fisrt First Etudiez
